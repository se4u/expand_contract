\documentclass[14pt]{article}
% ------ removeindent -------- %
\usepackage{parskip}
\setlength\parindent{0pt}
% -------------- ams ------------------- %
\usepackage{amssymb,amsmath,amsthm,mathtools,graphicx,url,xspace,booktabs,microtype,breakcites}
\newcommand{\union}{\cup}
\renewcommand{\th}[1]{$#1^{\textrm{th}}$}
\newcommand{\tr}[0]{\tilde{r}}
\newcommand{\tk}[0]{\tilde{k}}
\newcommand{\tM}[0]{\tilde{M}}
\newcommand{\tm}[0]{\tilde{m}}
\newcommand{\rmP}[0]{\mathrm{P}}
\newcommand{\rmK}[0]{\mathrm{K}}
\newcommand{\rmS}[0]{\mathrm{S}}
\title{\vspace{-6em}Expand Contract Notation}
\author{}
\date{}
% ------------------ listings ---------------------- %
\usepackage[]{listings}
\DeclareFixedFont{\ttb}{T1}{txtt}{bx}{n}{9} % for bold
\DeclareFixedFont{\ttm}{T1}{txtt}{m}{n}{9}  % for normal
% Defining colors
\usepackage{color}
\definecolor{deepblue}{rgb}{0,0,0.5}
\definecolor{deepred}{rgb}{0.6,0,0}
\definecolor{deepgreen}{rgb}{0,0.5,0}
% Python style for highlighting
\newcommand\pythonstyle{\lstset{
    language=Python,
    backgroundcolor=\color{white}
    basicstyle=\ttm,
    otherkeywords={self},
    keywordstyle=\ttb\color{deepblue},
    emph={MyClass,__init__},
    emphstyle=\ttb\color{deepred},
    stringstyle=\color{deepgreen},
    commentstyle=\color{red},
    frame=tb,
    showstringspaces=false
  }}

% Python environment
\lstnewenvironment{python}[1][]
{
  \pythonstyle
  \lstset{#1}
}
{}
\begin{document}
\maketitle
Let $\mathcal{P}$ be the set of all points and $\rmP$ be its cardinality.
Let $\mathcal{S}$ be the set of all seeds.
Let $\rmS$ be the size of $\mathcal{S}$.
We will use  $i$ to index the points and $j$ to index the seeds.
Let $d_{ij}$ be the distance of point $p_i \in \mathbb{R}^d$ from
seed $s_j \in \mathbb{R}^d$.
Let $\tr$ be the sequence of unique $d_{ij}$ values of length $\rmK$.
Let $r_k$ be the \th{k} radius value in $\tr$. We assume that $\tr$
is sorted in ascending order, i.e. $\tr_1 < \tr_2 < \ldots < \tr_{\rmK}$.\footnote{$\tr$ does not contain duplicates therefore it can be sorted into a strictly increasing sequence.}
Define $c_{ik}$ to be the number of seeds $s_j$, such that $d_{ij} \le r_k$.
Let $D \in \mathbb{R}^{\rmP \times \rmS}$ and
$C \in \mathbb{N}^{\rmP \times \rmK}$ be matrices containing
$d_{ij}$ and $c_{ik}$ respectively.

\paragraph{Naive Method:}
Our goal is to compute $C$ efficiently; the following naive method
for computing $C$ performs $O(\rmK\rmS\rmP)$ operations.
\begin{python}
  for k, r_k in enumerate(r_tilde):
      C[:, k] = (D > r_k).sum(axis=1)
\end{python}
Since $\rmK$ can be equal to $\mathrm{SP}$ therefore this method is impractical.

\paragraph{Fast Computation of $C$}
Let $m_{ij}$ be the index such that $d_{ij} > r_{m_{ij}}$ and $d_{ij} \le r_{m_{ij} + 1}$.
By construction, seed $j$ will contribute a value of 1 to $c_{ik}$ for all
$r_k \ge r_{m_{ij}}$, i.e. $c_{ik}  = \sum_{j \in \{1, \ldots, \rmP\}} \mathbb{I}[m_{ij} \le k]$. Clearly, when $k=\rmK$ then $c_{ik} = \rmS$.

Define $M \in \mathbb{N}^{\rmP \times \rmS}$ to be the matrix that contains
$m_{ij}$. This matrix can be computed in time $O(\rmS\rmP\log(\rmK))$ which is efficient even if $\rmK = \rmS \rmP$. After computing $M$ we can sort each
row of $M$ in the ascending order using $O(\rmP\rmS\log(\rmS))$
operations to create the matrix $\tM$.
$\tM$ is a compressed version of $C$. For example, if $k < \tm_{i1}$
then $c_{ik}$ equals $0$, or if $k \ge \tm_{i\rmS}$ then $c_{ik}$ equals
$\rmS$. Finally, if $\tm_{ij} \le k < \tm_{i (j+1)}$ then $c_{ik} = j$, therefore
$c_{ik}$ can be computed using a binary search on the rows of $\tM$.

\paragraph{Inflation Ranking}
The Inflation Ranking uses the following decision function:
$p_{i} > p_{i'}$ if there exists $k$ such that $c_{ik} > c_{i'k}$ and
$c_{i\tk} = c_{i'\tk}\ \forall \tk < k$. Assuming access to $\tM$ this
decision function can be computed efficiently by comparing the prefixes of
the sequences $\tm_{i:}$ and $\tm_{i':}$ and by finding the first prefix where
the two sequences differ. Let $j$ be the first index starting from $1$ such
that $\tm_{ij} \ne \tm_{i'j}$. If $\tm_{ij} > \tm_{i'j}$ then $p_{i} < p_{i'}$
otherwise $p_i > p_{i'}$.

% \bibliographystyle{plain}
% \bibliography{references.bib}
\end{document}


%%% Local Variables:
%%% mode: latex
%%% TeX-master: t
%%% End:
