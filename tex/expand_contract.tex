\documentclass[14pt]{article}
% ------ removeindent -------- %
\usepackage{parskip,listings}
\setlength\parindent{0pt}
% -------------- ams ------------------- %
\usepackage{amssymb,amsmath,amsthm,mathtools,graphicx,url,xspace,booktabs,microtype,breakcites}
\newcommand{\union}{\cup}
\renewcommand{\th}[1]{$#1^{\textrm{th}}$}
\newcommand{\td}[0]{\tilde{d}}
\title{Expand Contract Notation}
\author{PR}

\begin{document}
\maketitle

\section{Notation}
\label{sec:notation}


Let $\mathcal{S}$ be the set of all seeds and let $\mathrm{S}$ be the
number of seeds. We index the seeds using $i$. Let $\mathcal{P}$ be the
set of all points and $\mathrm{P}$ be the number of points. We index the
points using $j$. Let $d_{ij}$ be the distance of the \th{i} seed from the
\th{j} point. Let $s_i$ and $p_j$ be elements of $\mathbb{R}^d$.
Let $r_k$ be the \th{k} radius value from a length $\mathrm{K}$ series.
Let $c_{jk}$ be the number of seeds $s_i$ such that $d_{ij} < r_k$.
Let $D \in \mathbb{R}^{\mathrm{S} \times \mathrm{P}}$ and
$C \in \mathbb{N}^{\mathrm{P} \times \mathrm{K}}$ be matrices that contain $d_{jk}$
and $c_{jk}$ respectively.

Let $\td$ be the sequence of all unique $d_{ij}$ values sorted in descending order,
such that $\td_{k} > \td_{k+1}$. We will typically let $r_k = \td_{k}$.

We want to compute the matrix $C$ efficiently, the naive method for computing $C$ will
be as follows:
\begin{lstlisting}
  for k, r_k in enumerate(r_tilde):
      C[:, k] = (D > r_k).sum(axis=0)
\end{lstlisting}
However the above method has complexity $O(KSP)$.
Instead of using this method we want to reduce the complexity to a lower order.
\subsection{Fast Computation of $C$}
\label{ssec:fast}
Let $m_{ij}$ be the index such that $d_{ij} \ge r_{m_{ij}}$ and $d_{ij} < r_{m_{ij} + 1}$.
and let $M \in \mathbb{N}^{\mathrm{S} \times \mathrm{P}}$ be the matrix that contains
$m_{ij}$. This matrix can be computed in time $O(SP\log(K))$.
Then $c_{jk} = \sum_i {[m_{ij} > k]}$.
\footnote{The $[.]$ notation is the Iverson bracket.}
Let $\tilde{m}_j = \textrm{sort}(m_{ij})$. Computing $\tilde{m}_j$ only takes
$O(S\log(S))$ computations and the values of $c_{j\tilde{m}_{jk}}$ to $c_{j\tilde{m}_{j{(k+1)}}}$ are the same and they equal $k$. So $\tilde{m}_j$ is a compressed representation
of the \th{j} row of $C$.







\bibliographystyle{plain}
\bibliography{references.bib}

\end{document}


%%% Local Variables:
%%% mode: latex
%%% TeX-master: t
%%% End:
